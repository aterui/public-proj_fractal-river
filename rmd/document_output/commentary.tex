% Options for packages loaded elsewhere
\PassOptionsToPackage{unicode}{hyperref}
\PassOptionsToPackage{hyphens}{url}
%
\documentclass[
  12pt,
]{article}
\usepackage{amsmath,amssymb}
\usepackage{lmodern}
\usepackage{iftex}
\ifPDFTeX
  \usepackage[T1]{fontenc}
  \usepackage[utf8]{inputenc}
  \usepackage{textcomp} % provide euro and other symbols
\else % if luatex or xetex
  \usepackage{unicode-math}
  \defaultfontfeatures{Scale=MatchLowercase}
  \defaultfontfeatures[\rmfamily]{Ligatures=TeX,Scale=1}
  \setmainfont[]{Times New Roman}
\fi
% Use upquote if available, for straight quotes in verbatim environments
\IfFileExists{upquote.sty}{\usepackage{upquote}}{}
\IfFileExists{microtype.sty}{% use microtype if available
  \usepackage[]{microtype}
  \UseMicrotypeSet[protrusion]{basicmath} % disable protrusion for tt fonts
}{}
\makeatletter
\@ifundefined{KOMAClassName}{% if non-KOMA class
  \IfFileExists{parskip.sty}{%
    \usepackage{parskip}
  }{% else
    \setlength{\parindent}{0pt}
    \setlength{\parskip}{6pt plus 2pt minus 1pt}}
}{% if KOMA class
  \KOMAoptions{parskip=half}}
\makeatother
\usepackage{xcolor}
\usepackage[margin=1in]{geometry}
\usepackage{longtable,booktabs,array}
\usepackage{calc} % for calculating minipage widths
% Correct order of tables after \paragraph or \subparagraph
\usepackage{etoolbox}
\makeatletter
\patchcmd\longtable{\par}{\if@noskipsec\mbox{}\fi\par}{}{}
\makeatother
% Allow footnotes in longtable head/foot
\IfFileExists{footnotehyper.sty}{\usepackage{footnotehyper}}{\usepackage{footnote}}
\makesavenoteenv{longtable}
\usepackage{graphicx}
\makeatletter
\def\maxwidth{\ifdim\Gin@nat@width>\linewidth\linewidth\else\Gin@nat@width\fi}
\def\maxheight{\ifdim\Gin@nat@height>\textheight\textheight\else\Gin@nat@height\fi}
\makeatother
% Scale images if necessary, so that they will not overflow the page
% margins by default, and it is still possible to overwrite the defaults
% using explicit options in \includegraphics[width, height, ...]{}
\setkeys{Gin}{width=\maxwidth,height=\maxheight,keepaspectratio}
% Set default figure placement to htbp
\makeatletter
\def\fps@figure{htbp}
\makeatother
\setlength{\emergencystretch}{3em} % prevent overfull lines
\providecommand{\tightlist}{%
  \setlength{\itemsep}{0pt}\setlength{\parskip}{0pt}}
\setcounter{secnumdepth}{-\maxdimen} % remove section numbering
\newlength{\cslhangindent}
\setlength{\cslhangindent}{1.5em}
\newlength{\csllabelwidth}
\setlength{\csllabelwidth}{3em}
\newlength{\cslentryspacingunit} % times entry-spacing
\setlength{\cslentryspacingunit}{\parskip}
\newenvironment{CSLReferences}[2] % #1 hanging-ident, #2 entry spacing
 {% don't indent paragraphs
  \setlength{\parindent}{0pt}
  % turn on hanging indent if param 1 is 1
  \ifodd #1
  \let\oldpar\par
  \def\par{\hangindent=\cslhangindent\oldpar}
  \fi
  % set entry spacing
  \setlength{\parskip}{#2\cslentryspacingunit}
 }%
 {}
\usepackage{calc}
\newcommand{\CSLBlock}[1]{#1\hfill\break}
\newcommand{\CSLLeftMargin}[1]{\parbox[t]{\csllabelwidth}{#1}}
\newcommand{\CSLRightInline}[1]{\parbox[t]{\linewidth - \csllabelwidth}{#1}\break}
\newcommand{\CSLIndent}[1]{\hspace{\cslhangindent}#1}
\usepackage[left]{lineno}
\linenumbers
\ifLuaTeX
  \usepackage{selnolig}  % disable illegal ligatures
\fi
\IfFileExists{bookmark.sty}{\usepackage{bookmark}}{\usepackage{hyperref}}
\IfFileExists{xurl.sty}{\usepackage{xurl}}{} % add URL line breaks if available
\urlstyle{same} % disable monospaced font for URLs
\hypersetup{
  pdftitle={Mathematical characterization of fractal river networks},
  hidelinks,
  pdfcreator={LaTeX via pandoc}}

\title{Mathematical characterization of fractal river networks}
\author{}
\date{\vspace{-2.5em}}

\begin{document}
\maketitle

Author: Akira Terui\textsuperscript{1,*}

Email: hanabi0111@gmail.com

\textsuperscript{1}Department of Biology, University of North Carolina at Greensboro, Greensboro, NC 27410, USA

\hypertarget{abstract}{%
\section{Abstract}\label{abstract}}

River networks exhibit fractal characteristics. Quantifying river network complexity is crucial for unveiling the role of river fractals in driving riverine ecological dynamics, and researchers have used a metric of ``branching probability'' to do so. Previous studies showed that this metric reflects the fractal nature of river networks. However, a recent article by Carraro and Altermatt (2022) contradicted this classical observation and concluded that branching probability is ``scale dependent.'' I dispute this claim and argue that their major conclusion is derived merely from their misconception of scale invariance. Their analysis in the original article (Figure 3a) provided evidence that branching probability is scale invariant (i.e., branching probability exhibits a power-law scaling), although the authors erroneously interpreted this result as a sign of scale dependence. In this paper, I re-introduce the definition of scale invariance and show that branching probability meets this definition. This provided an opportunity to address the distinct use of ``scale invariance'' and ``scaling'' in the ecological literature.

\hypertarget{maintext}{%
\section{Maintext}\label{maintext}}

Rivers form complex branching networks with fractal characteristics -- ``there are basins within basins within basins, all of them looking alike (Rinaldo et al. 2014).'' The ecological implication of fractal structure has gained great interest over the past few decades (Grant et al. 2007; Carrara et al. 2012; Mari et al. 2014), and there have been concerted efforts to investigate the role of river fractals in controlling metapopulation (Yeakel et al. 2014; Mari et al. 2014; Terui et al. 2018) and metacommunity dynamics (Carrara et al. 2012; Anderson and Hayes 2018; Terui et al. 2021). The proper quantification of river network complexity is therefore crucial, and some researchers used a metric of ``branching probability,'' i.e., the probability of observing a joining tributary per unit river distance (Yeakel et al. 2014; Terui et al. 2018; Terui et al. 2021). It has been shown that this measure reflects the fractal nature of river networks because it exhibits the sign of scale invariance (Peckham and Gupta 1999; Rodríguez-Iturbe and Rinaldo 2001; Terui et al. 2018). However, a recent article by Carraro and Altermatt (2022) countered this classical observation and concluded that branching probability is ``scale dependent.'' They seriously criticized past research (Yeakel et al. 2014; Anderson and Hayes 2018; Terui et al. 2018; Terui et al. 2021) for the use of branching probability as a measure of river fractals, provocatively concluding ``\emph{an alleged property of such random networks (branching probability) is a scale dependent quantity that does not reflect any recognized metric of rivers' fractal character\ldots{}}'' in the Abstract. Here, I would like to communicate my concern that their major conclusion is merely a misconception of the term ``scale invariance.'' Indeed, the reanalysis of their data proved that branching probability is scale invariant. In what follows, I re-introduce the definition of scale invariance and show that branching probability meets this definition.

\hypertarget{premise-the-definition-of-scale-invariance}{%
\section{Premise: the definition of scale invariance}\label{premise-the-definition-of-scale-invariance}}

The concept of scale invariance is inherently confusing and is often misunderstood among ecologists. To clarify this concept, let me use the most famous example of a scale invariant object -- the complex structure of the coastline (Mandelbrot 1967). In Mandelbrot's pioneering work, he measured the coastline length \(y\) as multiples of a ruler with unit length \(x\) (the ``observation scale''). He observed that the length \(y\) diverged to reach infinity as \(x \rightarrow 0\), because finer but similar zig-zag patterns emerged over and over as he used a shorter ruler (Figure 1); that is, there was no specific scale that uniquely characterizes the coastline length (the ``lack of characteristic scale''; Table 1). However, when the coastline length was expressed as a function of scale \(x\) (\(y = f(x)\)), the function obeyed a simple rule that reflects the fractal nature of the observed object: the power law.

\[
\begin{aligned}
y = f(x) = cx^z &&\text{(1),}
\end{aligned}
\]

where \(c\) is the scaling constant and \(z\) is the scaling exponent. The scale invariance of this function can be easily proved by multiplying an arbitrary scaling factor \(\lambda\):

\[
\begin{aligned}
y'=f(\lambda x)=c\lambda^zx^z = \lambda^z f(x) = \lambda^z y  &&\text{(2),}
\end{aligned}
\]

The above equation is interpreted as a sign of scale invariance because the multiplicative extension/shrink of scale \(x\) by factor \(\lambda\) results in the same shape of the original object \(y\) but with a \emph{different scale} (Proekt et al. 2012) (i.e., the structural property or function is preserved). Hence, the observed object \(y\) (= function \(f(x)\)) is said to be scale invariant.

Although a power law is the most famous example of scale invariance, any function \(f(x)\) that suffices the following equation is said to be scale invariant (Rodríguez-Iturbe and Rinaldo 2001; Proekt et al. 2012):

\[
\begin{aligned}
f(\lambda x) = \lambda^z f(x) &&\text{(3),}
\end{aligned}
\]

A general property of this function is ``scaling'' (Table 1), in which the dimensional physical quantity of the object (e.g., length) changes predictably across scale \(x\) (Rodríguez-Iturbe and Rinaldo 2001).

Conversely, a ``scale dependent'' object possesses contrasting properties (Table 1). A square exemplifies a scale dependent object. Assuming \(500\) km on a side, the perimeter (\(500 \times 4 =2000\) km) remains constant at the divisors of \(500\) (\(x = 1, 2, 4, 5,...,500\)); otherwise, the measured perimeter becomes shorter than the true length if we disregard rulers that do not fit the square's side (\(y = x \lfloor \frac{500}{x} \rfloor \times 4\), where \(\lfloor \cdot \rfloor\) denotes the integer part of the division; see Figure 1). Therefore, the square has characteristic scales that uniquely identify its physical quantity. This property is called ``non-scaling,'' which contrasts with the property of a scale invariant object (Rodríguez-Iturbe and Rinaldo 2001). Although I used a square for the sake of simplicity, a more general definition of non-scaling is \(f(\lambda x) \ne \lambda^zf(x)\). Thus, ``scale dependence'' and ``scale invariance'' are contrasting observations.

\begin{longtable}[]{@{}ll@{}}
\caption{Key terminology}\tabularnewline
\toprule()
Object type & Property \\
\midrule()
\endfirsthead
\toprule()
Object type & Property \\
\midrule()
\endhead
Scale invariant & Scaling, the lack of characteristic scale \\
Scale dependent & Non-scaling, the existence of characteristic scale \\
\bottomrule()
\end{longtable}

\begin{figure}
\centering
\includegraphics{../output/figure_example.pdf}
\caption{\label{fig:example}Graphical comparison of scale invariant and scale dependent objects. Black lines indicate an example of a scale invariant object (coastline), while gray lines show an example of a scale dependent object (square). For the square, the solid gray lines indicate the measured length as multiples of the integer ruler length \(x\), with red points showing measurements at the divisors of 500 (the ``characteristic scale''). The dotted lines denote the true perimeter of the square. The left (a) and right (b) panels are ordinary and log-log plots, respectively.}
\end{figure}

\hypertarget{power-law-scaling-of-branching-probability}{%
\section{Power-law scaling of branching probability}\label{power-law-scaling-of-branching-probability}}

Here, I reanalyze the data in Carraro and Altermatt (2022) to assess the scale invariance of branching probability. By definition, branching probability is said to be scale invariant if it follows a power law of observation scale \(x\). Although the observation scale can take a variety of forms, a common metric in river geomorphology is the threshold value \(A_T\) that defines the minimum watershed area of the river channel. Since \(A_T\) measures stream size, we extract a subset of wider river channels as \(A_T\) increases (Figure 2A and 2B); in other words, the resolution of the river network becomes coarser. I extracted river networks of \(50\) watersheds that were used in Carraro and Altermatt with MERIT Hydro (Yamazaki et al. 2019) (pixel size: \(90~\text{m} \times 90~\text{m}\)).

I used \(20\) values of \(A_T\) {[}km\(^2\){]} (\(A_T=1,...,1000\) with an equal interval at a \(\log_{10}\) scale, but confined to \(A_T < \text{total watershed area}\) for small rivers), at which I estimated branching probability. Branching probability \(p\) {[}\(\text{km}^{-1}\){]} is the probability of observing a joining tributary per unit river distance. This quantity can be estimated from the probability distribution of link length \(L\) (i.e., the length of the river channel from one confluence to another, or one confluence to the outlet/upstream terminal). Typically, the link length \(L\) follows an exponential distribution, and branching probability can be calculated as a cumulative distribution function of the exponential distribution (Terui et al. 2018; Terui et al. 2021):

\[
\begin{aligned}
p = 1 - \exp(-\lambda)  &&\text{(4),}
\end{aligned}
\]

where \(\lambda\) is the rate parameter of an exponential distribution, which approximates the inverse of mean link length {[}unit: \(\text{km}^{-1}\){]} (\(\lambda \approx N_L / \sum L\), \(N_L\) is the number of links). To evaluate the power law of branching probability (\(p \approx cA_T^z\)), I fitted the following log-log linear model with robust regression:

\[
\begin{aligned}
\log_{10} p_{i} = \log_{10} c_{w(i)} + z\log_{10} A_{T,i} + \varepsilon_{i}  &&\text{(5),}
\end{aligned}
\]

where \(p_i\) is the branching probability (\(i\) represents a data point), \(c_{w(i)}\) is the watershed-specific scaling constant (\(w\) refers to a watershed), \(z\) the scaling exponent, and \(\varepsilon_i\) the error term that is properly weighted by Huber's function. I used robust regression analysis to account for outliers. Outliers were typical for coarse river networks with fewer links (i.e., large \(A_T\) values) because the estimation accuracy of the mean link length \(\lambda^{-1}\) depends on the number of links \(N_L\). The analysis provided strong support for the power-law scaling of branching probability (Figure 2C) with the estimated scaling exponent of \(z = -0.48 \pm 0.003\).

\begin{figure}
\centering
\includegraphics{../output/figure_scaling.pdf}
\caption{\label{fig:fig-pr}(a and b) Examples of extracted river network (Eel River, US). In panel (a), a fine-resolution river network was extracted with \(A_T = 12.7~\text{km}^2\). In panel (b), a coarse river network was extracted with \(A_T = 233.6~\text{km}^2\). (c) Log-log plot substantiates the power-law scaling of branching probability \(p\) along the axis of observation scale \(A_T\). Colors indicate rivers highlighted in the original article (Carraro and Altermatt 2022). Individual data points are shown in dots, whose transparency is proportional to the number of links \(N_L\) (i.e., the sample size). Lines are predicted values from the robust regression model.}
\end{figure}

\hypertarget{practical-issues}{%
\section{Practical issues}\label{practical-issues}}

It is noteworthy that the similar power-law scaling was reported in Carraro and Altermatt (2022) using the inverse of mean link length \(\lambda\) (referred to as ``branching ratio \(p_r\)'' in the original article; see Figure 3a and Equation 1 in Carraro and Altermatt (2022)). Nevertheless, the authors erroneously interpreted this result as a sign of ``scale dependence.'' In addition, they used the term ``scaling'' as if it describes a property of scale dependence (page 3, right column). The misuse of the terms is not a simple issue of terminology. Based on their misconception, they expanded a discussion over two pages and concluded ``\emph{We therefore conclude that branching probability is a non-descriptive property of a river network, which by no means describes its inherent branching character, and depends on the observational scale}'' (page 3). Hence, Carraro and Altermatt (2022) boldly used their misconception to invalidate past research that correctly referred to branching probability (or a cumulative distribution function of link length) as scale invariant (Peckham and Gupta 1999; Rodríguez-Iturbe and Rinaldo 2001; Moore et al. 2015; Terui et al. 2018; Terui et al. 2021).

Along this line, Carraro and Altermatt (2022) also argued that branching probability does not represent river's branching character (page 3 to page 4, left column) because of the lack of characteristic scale (i.e., the unique value of branching probability does not exist for a given river network). This is clearly incorrect. We just need to perform a proper rescaling (or non-dimensionalization) to capture the structure of scale-invariant objects (Rodríguez-Iturbe and Rinaldo 2001). Such rescaling is possible once the scaling equation is established:

\[
\begin{aligned}
p A_T^{-z} \approx c = constant &&\text{(6),}
\end{aligned}
\]

As evident from equation (6), we can obtain the unique value of (dimensionless) branching probability for a given river network after proper rescaling. This rescaling technique has been used for decades to characterize the physical/biological structure of scale invariant objects (Rodríguez-Iturbe and Rinaldo 2001; Brown et al. 2004; Rinaldo et al. 2014). In fact, previous studies used a unit scale of \(A_T\) (\(A_T = 1~\text{km}^2\); notice that \(p \approx c\) when \(A_T = 1\)) to approximate the rescaled branching probability (Terui et al. 2018; Terui et al. 2021). As such, Carraro and Altermatt (2022) provided no evidence that limits the use of branching probability as a measure of river network structure.

Overall, it is clear that the authors fundamentally misunderstood the mathematical definition of scale invariance. Inappropriate arguments arising from the misconception spanned from page 2 to 4; thus, correcting this misconception compromises the substantial portion of their article.

\hypertarget{distinctinct-terminology-in-ecology}{%
\section{Distinctinct terminology in ecology}\label{distinctinct-terminology-in-ecology}}

The misconception in Carraro and Altermatt (2022) may be rooted in the distinct use of the concerned terms in ecology. Although some ecologists follow the mathematical definitions (Hubbell 2001; Brown et al. 2002, 2004), the majority of research uses the terms in stark contrast with fractal theory. At least, there are three types of contrast.

First, ecologists use scale invariance when the observation is scale dependent. This is common in food web ecology and dates back to at least the 1980s. In a pioneering work on food web properties, Briand and Cohen (1984) found no statistical trends in the proportion of basal and consumer species in a food web over a wide range of total species richness (i.e., the regression coefficient of total species richness was not significantly different from zero). Based on this finding, they concluded that food web properties are ``scale invariant,'' as appeared in the article's title. The context shows that scale invariance in their article meant the constancy of the proportional abundance of certain trophic levels. Since then, this distinct terminology has been inherited in follow-up studies for approximately 40 years (Sugihara et al. 1989; Bersier et al. 1999; Galiana et al. 2021). For example, Galiana et al. (2021) investigated food web properties across spatial scales and wrote:

\begin{quote}
\ldots the proportion of species per trophic level and the proportion of overlap in the consumers' diet were largely \emph{scale-invariant} (Fig. 3). The proportion of basal, intermediate, and top species showed \emph{similar values from local to regional spatial scales}\ldots{} (emphasis added)
\end{quote}

Clearly, scale invariance is used to describe food web properties that do not change from local to regional scales in a statistical sense; this is what is called ``scale dependence'' (or ``non-scaling'') in fractal theory. If truly scale invariant in a fractal sense, the food web properties should change according to the power law scaling of observation scale (species richness or area).

Second, ecologists use scale dependence when the observation is scale invariant. Spatial scaling of biodiversity, such as species-area relationships (SARs), exemplifies this situation. Species richness typically obeys a power-law function of area, a relationship currently known as the Arrhenius SAR (Arrhenius 1921). This scaling feature enables us to predict species richness from small to large areas because a consistent ``scaling law'' exists across spatial scales. This property has contributed to the spatial design of aquatic and terrestrial protected areas (Neigel 2003; Desmet and Cowling 2004), which encompass more
than 30 million km\(^2\) globally (Deguignet et al. 2014). Nevertheless, ecologists refer to such scaling relationships as scale-dependent (Palmer and White 1994; Chase et al. 2018; Nishizawa et al. 2022) in contrast to the common terminology in fractal theory. Only a few ecologists refer to power-law SARs as scale invariant following the definition of fractal theory (e.g., Hubbell 2001).

Lastly, ecologists use scaling when scaling is impossible. Countless numbers of ecological research use the term scaling when observations do not follow the scaling relationship defined in equation (3) (Keil et al. 2018; Jarzyna and Jetz 2018; Gonzalez et al. 2020; Galiana et al. 2021). For example, Keil et al. (2018) related species extirpation events to area and described the observed non-monotonic relationships as scaling; however, this is what is called ``non-scaling'' in fractal theory. The ``ecological'' use of scaling can be inferred from contexts if there is no mention to fractals or scaling theory. Yet, it is difficult to ``adjust'' the meaning of the term when some researchers discuss this ecological term of scaling with explicit reference to scaling theory (Gonzalez et al. 2020), as scaling theory defines scaling in a completely opposite manner (Mandelbrot 1967; Rodríguez-Iturbe and Rinaldo 2001; Brown et al. 2002, 2004; Proekt et al. 2012; Rinaldo et al. 2014). Such a description is self-contradicting for those who have backgrounds in fractal/scaling theory.

The divergent use of the same terms among disciplines could be a natural outcome given the changing nature of language. One may argue that this is not a serious problem because ``\emph{people are skilled at deciphering meaning from context}'' (Hodges 2008). I agree, but only if the central idea of the concept remains similar. For example, the term ``metapopulation'' is defined differently among studies (Hanski and Gilpin 1991; Fronhofer et al. 2012; Poos and Jackson 2012; Terui et al. 2018), but their ``operational'' definitions share the core of the concept. This is not the case with scale invariance. In ecology, scale invariance (and scaling) is used in a way that completely undermines the original theoretical implications of fractals. In an extreme case, researchers describe that a power function is an observation that does not satisfy scaling laws (Havens 1992). It is likely that Carraro and Altermatt (2022) -- despite their clear focus on fractals -- picked this ecological conceptualization of ``scale invariance'' and ``scaling'' to invalidate past research that followed the mathematical definition of the same terms. An unsolved mystery, however, is that the authors followed the correct mathematical definitions in their previous publications (Giometto et al. 2013; Carraro et al. 2020).

\hypertarget{concluding-remarks}{%
\section{Concluding remarks}\label{concluding-remarks}}

Scaling is a fundamental property of scale-invariant observations, allowing for generalizations across different scales (Levin 1992; Clark et al. 2021). This concept plays a central role in ecological research. However, the meanings of key terms have diverged within the field of ecology. In contrast, the same terms in fractal theory have retained their mathematically defined nature (equation (3) and Table 1) since their initial introduction in the 1960s (Mandelbrot 1967; Rodríguez-Iturbe and Rinaldo 2001; Brown et al. 2002, 2004; Proekt et al. 2012; Rinaldo et al. 2014).

The gulf between ecology and fractal theory is deep, and achieving consistent terminology may prove challenging. As such, it is highly recommended that authors clarify the context of terminology in their research to promote better understanding. Simultaneously, it is crucial to encourage ecologists to recognize the mathematical definition of scale invariance. By fostering clearer communication and understanding between the fields, we can pave the way for more productive interdisciplinary collaborations. Otherwise, the divergent terminology creates unnecessary debates that hinder scientific advancement. The disrespectful arguments presented in Carraro and Altermatt (2022) serve as an unfortunate example of the negative consequences resulting from this divergence.

\hypertarget{data-availability}{%
\subsection{Data availability}\label{data-availability}}

Codes and data are available at \url{https://github.com/aterui/public-proj_fractal-river}.

\hypertarget{competing-interest}{%
\subsection{Competing interest}\label{competing-interest}}

None declared.

\hypertarget{acknowledgements}{%
\subsection{Acknowledgements}\label{acknowledgements}}

I thank Ryosuke Iritani and Shota Shibasaki for their comments on the earlier version of this manuscript. This material is based upon work supported by the National Science Foundation through the Division of Environmental Biology (DEB 2015634).

\pagebreak

\hypertarget{references}{%
\subsection*{References}\label{references}}
\addcontentsline{toc}{subsection}{References}

\hypertarget{refs}{}
\begin{CSLReferences}{1}{0}
\leavevmode\vadjust pre{\hypertarget{ref-andersonEffectsDispersalRiver2018}{}}%
Anderson KE, Hayes SM (2018) The effects of dispersal and river spatial structure on asynchrony in consumer\textendash resource metacommunities. Freshwater Biology 63:100--113. \url{https://doi.org/10.1111/fwb.12998}

\leavevmode\vadjust pre{\hypertarget{ref-arrheniusSpeciesArea1921}{}}%
Arrhenius O (1921) Species and area. Journal of Ecology 9:95--99. \url{https://doi.org/10.2307/2255763}

\leavevmode\vadjust pre{\hypertarget{ref-bersierScaleInvariantScale1999}{}}%
Bersier L-F, Dixon P, Sugihara G (1999) Scale-invariant or scale-dependent behavior of the link density property in food webs: A matter of sampling effort? The American Naturalist 153:676--682. \url{https://doi.org/10.1086/303200}

\leavevmode\vadjust pre{\hypertarget{ref-briandCommunityFoodWebs1984}{}}%
Briand F, Cohen JE (1984) Community food webs have scale-invariant structure. Nature 307:264--267. \url{https://doi.org/10.1038/307264a0}

\leavevmode\vadjust pre{\hypertarget{ref-brownMetabolicTheoryEcology2004}{}}%
Brown JH, Gillooly JF, Allen AP, et al (2004) Toward a metabolic theory of ecology. Ecology 85:1771--1789. \url{https://doi.org/10.1890/03-9000}

\leavevmode\vadjust pre{\hypertarget{ref-brownFractalNatureNature2002}{}}%
Brown JH, Gupta VK, Li B-L, et al (2002) The fractal nature of nature: Power laws, ecological complexity and biodiversity. Philosophical Transactions of the Royal Society of London Series B: Biological Sciences 357:619--626. \url{https://doi.org/10.1098/rstb.2001.0993}

\leavevmode\vadjust pre{\hypertarget{ref-carraraDendriticConnectivityControls2012}{}}%
Carrara F, Altermatt F, Rodriguez-Iturbe I, Rinaldo A (2012) Dendritic connectivity controls biodiversity patterns in experimental metacommunities. Proceedings of the National Academy of Sciences 109:5761--5766. \url{https://doi.org/10.1073/pnas.1119651109}

\leavevmode\vadjust pre{\hypertarget{ref-carraroOptimalChannelNetworks2022}{}}%
Carraro L, Altermatt F (2022) Optimal {Channel Networks} accurately model ecologically-relevant geomorphological features of branching river networks. Communications Earth \& Environment 3:1--10. \url{https://doi.org/10.1038/s43247-022-00454-1}

\leavevmode\vadjust pre{\hypertarget{ref-carraroGenerationApplicationRiver2020}{}}%
Carraro L, Bertuzzo E, Fronhofer EA, et al (2020) Generation and application of river network analogues for use in ecology and evolution. Ecology and Evolution 10:7537--7550. \url{https://doi.org/10.1002/ece3.6479}

\leavevmode\vadjust pre{\hypertarget{ref-chaseEmbracingScaledependenceAchieve2018}{}}%
Chase JM, McGill BJ, McGlinn DJ, et al (2018) Embracing scale-dependence to achieve a deeper understanding of biodiversity and its change across communities. Ecology Letters 21:1737--1751. \url{https://doi.org/10.1111/ele.13151}

\leavevmode\vadjust pre{\hypertarget{ref-clarkGeneralStatisticalScaling2021}{}}%
Clark AT, Arnoldi J-F, Zelnik YR, et al (2021) General statistical scaling laws for stability in ecological systems. Ecology Letters n/a: \url{https://doi.org/10.1111/ele.13760}

\leavevmode\vadjust pre{\hypertarget{ref-deguignet2014UnitedNations2014}{}}%
Deguignet M, Juffe-Bignoli D, Harrison J, et al (2014) 2014 {United Nations List} of {Protected Areas}. UNEP-WCMC: Cambridge, UK

\leavevmode\vadjust pre{\hypertarget{ref-desmetUsingSpeciesareaRelationship2004}{}}%
Desmet P, Cowling R (2004) Using the species-area relationship to set baseline targets for conservation. Ecology and Society 9:art11. \url{https://doi.org/10.5751/ES-01206-090211}

\leavevmode\vadjust pre{\hypertarget{ref-fronhoferWhyAreMetapopulations2012}{}}%
Fronhofer EA, Kubisch A, Hilker FM, et al (2012) Why are metapopulations so rare? Ecology 93:1967--1978

\leavevmode\vadjust pre{\hypertarget{ref-galianaSpatialScalingFood2021}{}}%
Galiana N, Barros C, Braga J, et al (2021) The spatial scaling of food web structure across {European} biogeographical regions. Ecography 44:653--664. \url{https://doi.org/10.1111/ecog.05229}

\leavevmode\vadjust pre{\hypertarget{ref-giomettoScalingBodySize2013}{}}%
Giometto A, Altermatt F, Carrara F, et al (2013) Scaling body size fluctuations. Proceedings of the National Academy of Sciences 110:4646--4650. \url{https://doi.org/10.1073/pnas.1301552110}

\leavevmode\vadjust pre{\hypertarget{ref-gonzalezScalingupBiodiversityecosystemFunctioning2020}{}}%
Gonzalez A, Germain RM, Srivastava DS, et al (2020) Scaling-up biodiversity-ecosystem functioning research. Ecology Letters 23:757--776. \url{https://doi.org/10.1111/ele.13456}

\leavevmode\vadjust pre{\hypertarget{ref-grantLivingBranchesPopulation2007}{}}%
Grant EHC, Lowe WH, Fagan WF (2007) Living in the branches: Population dynamics and ecological processes in dendritic networks. Ecology Letters 10:165--175

\leavevmode\vadjust pre{\hypertarget{ref-hanskiMetapopulationDynamicsBrief1991}{}}%
Hanski I, Gilpin M (1991) Metapopulation dynamics: Brief history and conceptual domain. Biological Journal of the Linnean Society 42:3--16. \url{https://doi.org/10.1111/j.1095-8312.1991.tb00548.x}

\leavevmode\vadjust pre{\hypertarget{ref-havensScaleStructureNatural1992}{}}%
Havens K (1992) Scale and structure in natural food webs. Science 257:1107--1109. \url{https://doi.org/10.1126/science.257.5073.1107}

\leavevmode\vadjust pre{\hypertarget{ref-hodgesDefiningProblemTerminology2008}{}}%
Hodges KE (2008) Defining the problem: Terminology and progress in ecology. Frontiers in Ecology and the Environment 6:35--42. \url{https://doi.org/10.1890/060108}

\leavevmode\vadjust pre{\hypertarget{ref-hubbellUnifiedNeutralTheory2001}{}}%
Hubbell SP (2001) The {Unified Neutral Theory} of {Biodiversity} and {Biogeography}. {Princeton University Press}

\leavevmode\vadjust pre{\hypertarget{ref-jarzynaTaxonomicFunctionalDiversity2018}{}}%
Jarzyna MA, Jetz W (2018) Taxonomic and functional diversity change is scale dependent. Nature Communications 9:2565. \url{https://doi.org/10.1038/s41467-018-04889-z}

\leavevmode\vadjust pre{\hypertarget{ref-keilSpatialScalingExtinction2018}{}}%
Keil P, Pereira HM, Cabral JS, et al (2018) Spatial scaling of extinction rates: {Theory} and data reveal nonlinearity and a major upscaling and downscaling challenge. Global Ecology and Biogeography 27:2--13. \url{https://doi.org/10.1111/geb.12669}

\leavevmode\vadjust pre{\hypertarget{ref-levinProblemPatternScale1992}{}}%
Levin SA (1992) The problem of pattern and scale in ecology: The {Robert H}. {Macarthur} award lecture. Ecology 73:1943--1967. \url{https://doi.org/10.2307/1941447}

\leavevmode\vadjust pre{\hypertarget{ref-mandelbrotHowLongCoast1967}{}}%
Mandelbrot B (1967) How long is the coast of {Britain}? {Statistical} self-similarity and fractional dimension. Science 156:636--638. \url{https://doi.org/10.1126/science.156.3775.636}

\leavevmode\vadjust pre{\hypertarget{ref-mariMetapopulationPersistenceSpecies2014}{}}%
Mari L, Casagrandi R, Bertuzzo E, et al (2014) Metapopulation persistence and species spread in river networks. Ecology Letters 17:426--34. \url{https://doi.org/10.1111/ele.12242}

\leavevmode\vadjust pre{\hypertarget{ref-mooreEmergentStabilityLarge2015}{}}%
Moore JW, Beakes MP, Nesbitt HK, et al (2015) Emergent stability in a large, free-flowing watershed. Ecology 96:340--347

\leavevmode\vadjust pre{\hypertarget{ref-neigelSpeciesareaRelationshipsMarine2003}{}}%
Neigel JE (2003) Species-area relationships and marine conservation. Ecological Applications 13:138--145. \url{https://doi.org/10.1890/1051-0761(2003)013\%5B0138:SARAMC\%5D2.0.CO;2}

\leavevmode\vadjust pre{\hypertarget{ref-nishizawaLatitudinalGradientPlant2022}{}}%
Nishizawa K, Shinohara N, Cadotte MW, Mori AS (2022) The latitudinal gradient in plant community assembly processes: {A} meta-analysis. Ecology Letters 25:1711--1724. \url{https://doi.org/10.1111/ele.14019}

\leavevmode\vadjust pre{\hypertarget{ref-palmerScaleDependenceSpeciesarea1994}{}}%
Palmer MW, White PS (1994) Scale dependence and the species-area relationship. The American Naturalist 144:717--740. \url{https://doi.org/10.1086/285704}

\leavevmode\vadjust pre{\hypertarget{ref-peckhamReformulationHortonLaws1999}{}}%
Peckham SD, Gupta VK (1999) A reformulation of {Horton}'s laws for large river networks in terms of statistical self-similarity. Water Resources Research 35:2763--2777

\leavevmode\vadjust pre{\hypertarget{ref-poosImpactSpeciesspecificDispersal2012}{}}%
Poos MS, Jackson DA (2012) Impact of species-specific dispersal and regional stochasticity on estimates of population viability in stream metapopulations. Landscape Ecology 27:405--416

\leavevmode\vadjust pre{\hypertarget{ref-proektScaleInvarianceDynamics2012}{}}%
Proekt A, Banavar JR, Maritan A, Pfaff DW (2012) Scale invariance in the dynamics of spontaneous behavior. Proceedings of the National Academy of Sciences 109:10564--10569. \url{https://doi.org/10.1073/pnas.1206894109}

\leavevmode\vadjust pre{\hypertarget{ref-rinaldoEvolutionSelectionRiver2014}{}}%
Rinaldo A, Rigon R, Banavar JR, et al (2014) Evolution and selection of river networks: {Statics}, dynamics, and complexity. Proceedings of the National Academy of Sciences 111:2417--2424. \url{https://doi.org/10.1073/pnas.1322700111}

\leavevmode\vadjust pre{\hypertarget{ref-rodriguez-iturbeFractalRiverBasins2001}{}}%
Rodríguez-Iturbe I, Rinaldo A (2001) Fractal {River Basins}: {Chance} and {Self-Organization}. {Cambridge University Press}, {New York, NY}

\leavevmode\vadjust pre{\hypertarget{ref-sugiharaScaleInvarianceFood1989}{}}%
Sugihara G, Schoenly K, Trombla A (1989) Scale invariance in food web properties. Science 48--52

\leavevmode\vadjust pre{\hypertarget{ref-teruiMetapopulationStabilityBranching2018}{}}%
Terui A, Ishiyama N, Urabe H, et al (2018) Metapopulation stability in branching river networks. Proceedings of the National Academy of Sciences 115:E5963--E5969. \url{https://doi.org/10.1073/pnas.1800060115}

\leavevmode\vadjust pre{\hypertarget{ref-teruiEmergentDualScaling2021}{}}%
Terui A, Kim S, Dolph CL, et al (2021) Emergent dual scaling of riverine biodiversity. Proceedings of the National Academy of Sciences 118:e2105574118. \url{https://doi.org/10.1073/pnas.2105574118}

\leavevmode\vadjust pre{\hypertarget{ref-yamazakiMERITHydroHighresolution2019}{}}%
Yamazaki D, Ikeshima D, Sosa J, et al (2019) {MERIT} hydro: A high-resolution global hydrography map based on latest topography dataset. Water Resources Research 55:5053--5073. \url{https://doi.org/10.1029/2019wr024873}

\leavevmode\vadjust pre{\hypertarget{ref-yeakelSynchronisationStabilityRiver2014}{}}%
Yeakel JD, Moore JW, Guimarães PR, de Aguiar MAM (2014) Synchronisation and stability in river metapopulation networks. Ecology Letters 17:273--283. \url{https://doi.org/10.1111/ele.12228}

\end{CSLReferences}

\end{document}
